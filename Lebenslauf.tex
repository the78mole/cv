%% start of file `template.tex'.
%% Copyright 2006-2010 Xavier Danaux (xdanaux@gmail.com).
%
% This work may be distributed and/or modified under the
% conditions of the LaTeX Project Public License version 1.3c,
% available at http://www.latex-project.org/lppl/.


\documentclass[sans,11pt,a4paper]{moderncv}

% moderncv themes
%\moderncvtheme[blue]{casual}                % optional argument are 'blue' (default), 'orange', 'red', 'green', 'grey' and 'roman' (for roman fonts, instead of sans serif fonts)
%\moderncvtheme[green]{classic}              % idem
%\moderncvtheme[grey]{classic}               % idem
%\moderncvtheme[blue]{casual}                % idem
\moderncvtheme[blue]{classic}                % idem
\moderncvicons{marvosym}

% character encoding
\usepackage[ngerman]{babel}
%\usepackage[T1]{fontenc}
\usepackage[utf8]{inputenc}

% adjust the page margins
\usepackage[scale=0.8]{geometry}
%\setlength{\hintscolumnwidth}{3cm}						% if you want to change the width of the column with the dates
%\AtBeginDocument{\setlength{\maketitlenamewidth}{6cm}} % only for the classic theme, if you want to change the width of your name placeholder (to leave more space for your address details
\AtBeginDocument{\recomputelengths}                     % required when changes are made to page layout lengths

% personal data
\firstname{Daniel}
\familyname{Glaser}
\title{Lebenslauf}                      % optional, remove the line if not wanted
\address{Birkenweg 12}{91058 Erlangen}  % optional, remove the line if not wanted
\mobile{+49 176 21103885}               % optional, remove the line if not wanted
\phone{+49 9131 9333382}                % optional, remove the line if not wanted
%\fax{fax (optional)}                   % optional, remove the line if not wanted
\email{glaser@chaintronics.com}         % optional, remove the line if not wanted
%\homepage{homepage (optional)}         % optional, remove the line if not wanted
\extrainfo{geb. 20.12.1978, Werneck}    % optional, remove the line if not wanted
\photo[140pt]{daniel-profil}            % '64pt' is the height the picture must be resized to and 'picture' is the name of the picture file; optional, remove the line if not wanted

% to show numerical labels in the bibliography; only useful if you make citations in your resume
\makeatletter
%\renewcommand*{\bibliographyitemlabel}{\@biblabel{\arabic{enumiv}}}
\makeatother

% bibliography with mutiple entries
%\usepackage{multibib}
%\newcites{book,misc}{{Books},{Others}}

%\nopagenumbers{}                             % uncomment to suppress automatic page numbering for CVs longer than one page
%----------------------------------------------------------------------------------
%            content
%----------------------------------------------------------------------------------
\begin{document}
\maketitle

\section{Berufserfahrung}

\cventry{seit 03/23}{Senior Expert}{Angestellter}{Erlangen}{Elektrobit Automotive GmbH}%
{\begin{itemize}
		\item Aufbau und Betrieb von Entwicklungshardware für Fernzugriff (Embedded Linux)
		\item Implementierung und Wartung einer Fernverwaltungssoftware für Remote-Targets
		\item Automatisierung von Entwicklungsprozessen mit GitHub Actions
		\item Erstellung von Devcontainer Features und Images
		\item Automatisierung der IT-Umgebung mit Ansible
		\item Aufbau von Demonstratoren für Kundenpräsentationen und Messen
		\item Technische Vertriebsunterstützung: Besuch von Kundenevents
	\end{itemize}}

\cventry{seit 05/25}{Gesellschafter-Geschäftsführer}{}{Erlangen}{Elektro-Glaser GmbH}%
{\begin{itemize}
		\item Kaufmännische und technische Leitung des Unternehmens
		\item Elektroinstallation
		      \begin{itemize}
			      \item Installation, Wartung und Reparatur elektrischer Anlagen in Wohn- und Gewerbegebäuden
			      \item Installation und Anmeldung von Ladeeinrichtungen für Elektrofahrzeuge
			      \item Inbetriebnahme und Anmledung von Ladeeinrichtungen, Photovoltaikanlagen und Batteriespeichern
		      \end{itemize}
		\item Beratung speziell zu Elektromobilität, Photovoltaik und Energiemanagement
		\item Automatisierung von Geschäftsprozessen
	\end{itemize}}

\cventry{seit 04/18}{Gesellschafter-Geschäftsführer}{}{Erlangen}{BizzMark UG (haftungsbeschränkt)}%
{\begin{itemize}
		\item Kaufmännische und technische Leitung des Unternehmens
		\item Beratung in den Bereichen
		      \begin{itemize}
			      \item ERP, Prozessautomatisierung, IT-Systeme (Start-Ups \& kleinere Unternehmen)
			      \item Embedded-Produktentwicklung (technische \& rechtliche Anforderungen)
			      \item Agilität im Bereich Business Development und Embeeded-Entwicklung
			      \item Architektur und Test von Embedded Hard- \& Software
		      \end{itemize}
		\item Entwicklungsdienstleistungen Hard-, Soft- \& Firmware, Web-Applikationen \& Backend
	\end{itemize}}

\cventry{08/21-02/23}{Projektleiter Produktentwicklung}{Angestellter}{Nürnberg}{Landis+Gyr GmbH}%
{\begin{itemize}
		\item Leitung von Wärmemengenzähler-Entwicklungsprojekten
		\item Leitung von Backend-Projekten (z.B. PKI)
		\item Schnittstelle zu Dienstleistern/Lieferanten (externe Entwicklung)
		\item Schnittstelle zu Entwicklungspartnern (gemeinsame Kundenprojekte)
		\item Interner Ansprechpartner Software-Tooling (Projektplanung)
	\end{itemize}}

\cventry{05/18--07/21}{Teamleiter \& Embeddedentwickler}{Angestellter}{Fürth}{Dommel GmbH}%
{\begin{itemize}
		\item Aufbau des Standorts Fürth
		\item Leitung des Entwicklungsteams
		\item Planung \& Systementwurf von Embedded-Projekten
		\item Hardware- \& Softwareentwicklung von eingebetteten Systemen (STM32, AVR, Holtek)
		\item Anwendungsentwicklung von PC-Applikationen (Firmware-Updater, Service-Tools; JavaSE, QT)
		\item Anforderungsanalyse \& -management in Embedded-Entwicklungs-Projekten
		\item Projektleitung \& Koordination der Entwicklungspartner
		\item Planung \& Aufbau einer CI/CD-IT-Infrastruktur für Embedded-Projekte (Jenkins, JIRA, GIT)
	\end{itemize}}

\cventry{03/17--03/18}{Geschäftsleiter}{Angestellter}{Nürnberg}{IPN Solutions GmbH \& Co. KG}%
{\begin{itemize}
		\item Geschäftsleitung in den Bereichen Finanzen, Marketing, Vertrieb, IKT, Akquise, Recruiting
		\item Geschäftsjahresplanung, Personalführung, Controlling und Bankenkommunikation
		\item Planung von In-House Kundenprojekten
		\item Beratung zur Unternehmensgründung, Produktplanung und Anforderungsanalyse
		\item Beratung im Bereich Digitalisierung, agiler Organisation und IT-Infrastruktur
		\item In-House Prozessautomatisierung mit Camunda (JavaEE)
		\item Beratung im Bereich agile Software- und Produktentwicklung
		\item Beratung bei der technischen Umsetzung von eingebetteten Systemen
		\item Planung und Betreuung der Geschäfts-IT
	\end{itemize}}

\cventry{07/14--02/17}{Senior Software Consultant}{Angestellter}{Nürnberg}{IPN Solutions GmbH \& Co. KG}%
{\begin{itemize}
		\item Teilprojektleitung \& SW-Architektur, \glqq digitale Straßenkarte\grqq\ (C++, Embedded Linux, ASIL-B)
		\item Aufbau einer CI-Toolchain für Embedded System \glqq digitale Straßenkarte\grqq\ (Jenkins, JUnit)
		\item Einführung agiler Entwicklungs-Methoden für Projekt \glqq digitale Straßenkarte\grqq\ (SCRUM)
		\item Entwicklung eines CA-Frontends für Verwaltung und Erstellung von Zertifikaten (Bootstrap)
		\item Aufbau eines Cert-Service für Device-Zertifikate (JavaEE, Bouncycastle, REST, WebUI)
		\item Erweiterung des Backends eines KfZ-Ferndiagnosesystems (JavaEE)
		\item Refactoring des bestehenden Backends eines KfZ-Ferndiagnosesystems (JavaEE)
	\end{itemize}}

\cventry{11/11--06/14}{Geschäftsführer}{Gesellschafter}{Erlangen}{beECO GmbH}%
{\begin{itemize}
		\item System- und Hardwareentwicklung, IT-Infrastruktur, Marketing \& Vertrieb
		\item Entwicklung eines verteilten Systems mit folgenden Komponenten
		      \begin{itemize}
			      \item Konfiguration einer Zentraleinheit (Embedded Industrial PC, Linux)
			      \item Implementierung der OSGi-Services für KWK-Optimierung (Java, OSGi/Equinox)
			      \item Erfassung von KWK-Kraftwerksdaten (Embedded uC Knoten , Ethernet, PoE)
			      \item Prognose der Folgetage auf Basis von Wetterdaten und KNN (Java, Encog)
			      \item Integration des Simulationskerns aus studentischer Arbeit (Java, OSGi/Equinox)
			      \item VPN zur Fernwartung und Verbindung mit Backend (OpenVPN)
			      \item Aufbau der zentralen Server-Infrastruktur (ETL z.B. GRIB-Wetterdaten; JavaEE)
			      \item Over-The-Air-Update aller Komponenten (AVR ETH-Bootloader, OSGi-Services, Linux-Basis)
		      \end{itemize}
	\end{itemize}}

\cventry{03/12--02/14}{Angestellter}{Wissenschaftlicher Mitarbeiter, Stipendiat}{Erlangen}{Lehrstuhl für Sensorik LSE, Friedrich-Alexander-Universität Erlangen-Nürnberg}%
{\begin{itemize}
		\item Förderprogramm FLÜGGE-Bayern (Existenzgründung, Projekt beECO)
		\item Erweiterung des Prototypen für Energiemanagement (Java, OSGi/Equinox)
		\item Betreuung von Studentischen Arbeiten
		      \begin{itemize}
			      \item Implementierung eines Event-basierten Simulationskerns (Java)
		      \end{itemize}
	\end{itemize}}

\cventry{03/11--02/12}{Stipendiat}{Selbstständiger}{Erlangen}{Lehrstuhl für Sensorik LSE, Friedrich-Alexander-Universität Erlangen-Nürnberg}%
{\begin{itemize}
		\item EXIST-Gründerstipendium (Projekt ambience, später beECO)
		\item Implementierung eines Prototypen für Energiemanagement (Java, OSGi/Equinox)
	\end{itemize}}

\cventry{12/07--12/10}{Angestellter}{Wissenschaftlicher Mitarbeiter, Doktorand}{Erlangen}{Lehrstuhl für Zuverlässige Schaltungen und Systeme LZS (ehem. LRS), Friedrich-Alexander-Universität Erlangen-Nürnberg}%
{\begin{itemize}
		\item Forschung im Projekt OKTOPUS
		      \begin{itemize}
			      \item Organisation des Arbeitspakets Ressourcenplanung und Modellierung
			      \item Entwicklung eines Konzepts zur systemunabhängigen Testbeschreibung
			      \item Implementierung einer Bibliothek zur Verarbeitung von ATML (Java)
			      \item Prototyperstellung für Ressourcenplanung und Testprogrammsynthese (Java)
			      \item Implementierung von generischen Instrumententreibern (LabView)
			      \item Modellierung \& Simulation: Prüfling, Schnittstellen, Testsystem (SystemC-AMS, VHDL-AMS)
		      \end{itemize}
		\item Lehre
		      \begin{itemize}
			      \item Betreuung von neun Studien-, Diplom- und Bachelorarbeiten
			      \item Betreuung des Modellierungs- und Simulations-Praktikums
		      \end{itemize}
	\end{itemize}}

\cventry{07/03--10/11}{Einzelunternehmer}{Selbstständiger}{Erlangen}{}{Konzeptionierung und Entwicklung von Hard- und Software%
	\begin{itemize}
		\item FPGA-Entwicklung in VHDL und zugehöriger Hardware
		\item Entwicklung eingebetteter Mikrocontrollersysteme (AVR)
		\item Entwurf von analogen, digitalen und mixed-signal Schaltungen
		\item Konzeptionierung und Implementierung vernetzter Mikrocontroller-Systeme
		\item Einrichtung eingebetteter Linux-Systeme
		      \begin{itemize}
			      \item VPN-Gateway (OpenVPN)
			      \item Datenerfassung und -speicherung
			      \item Grafische Benutzerschnittstellen
		      \end{itemize}
		\item Wartung und Inbetriebnahme bühnentechnischer Anlagen
		\item Betreuung von Linux-Servern
	\end{itemize}}

\cventry{06/07--10/07}{Student}{Wissenschaftliche Hilfskraft}{Erlangen}{Lehrstuhl für Rechnergestützten Schaltungsentwurf LRS, Friedrich-Alexander-Universität Erlangen-Nürnberg}{%
	\begin{itemize}
		\item Aufsetzen eines Halbleitertestpraktikums
		      \begin{itemize}
			      \item Entwicklung einer Hardware (Analogfrontend, ADC/DAC, Power, FPGA, Mikrocontroller)
			      \item VHDL-Entwicklung zur Emulation verschiedener Defekte in AD-Umsetzer
			      \item Programmierung des eingebetteten Mikrocontrollers zur Steuerung des FPGA
			      \item Analog- und Digital-Schaltungsentwurf
			      \item Entwicklung eines Testprogramms auf Credence SZ Testsystem
		      \end{itemize}
	\end{itemize}}

\cventry{06/04--05/07}{Student}{Wissenschaftliche Hilfskraft}{Erlangen}{Fraunhofer IIS, Bildsensorik}{Entwicklung von Hard- und Software, VHDL%
	\begin{itemize}
		\item FPGA-Entwicklung in VHDL und zugehöriger Hardware
		\item Programmierung eingebetteter Mikrocontrollersysteme (ARM, PowerPC)
		\item Algorithmenentwicklung in C und Implementierung in VHDL
		\item Analog- und Digital-Schaltungsentwurf
	\end{itemize}}

\cventry{03/02--08/03}{Geschäftsführer, Softwareentwickler}{Gesellschafter}{Lohr a. Main}{Powercontrols GbR}{Verkauf von PC-Hardwarekomponenten%
	\begin{itemize}
		\item Geschäftsführung -- Technik und Einkauf
		\item Erweiterung des Shopsystems osCommerce zur Anbindung von Großhändlern (PHP)
		\item Erweiterung des Shopsystems um einfache Warenwirtschaft (PHP)
		\item Automatisierte, stündliche Preis- und Verfügbarkeits-Aktualisierung (ETL, $>$25.000 Artikel)
	\end{itemize}}

%\cventry{2000--2003 (Ferienzeiten)}{Ferienarbeiter}{Facharbeiter, Energieelektroniker}{Lohr}{Bosch Rexroth AG}%
%{\begin{itemize}
%	\item Bühnentechnische Steuerungen aufbauen und testen
%	\item Wartung, Reperatur und Inbetriebnahme bühnentechnischer Anlagen
%\end{itemize}}

\pagebreak

%%%%%%%%%%%%%%%%%%
\section{Diplomarbeit}
\cvline{Titel}{\emph{Entwurf einer eingebetteten Architektur zur generischen und quelloffenen Dekodierung von H.264 und HE-AAC Datenströmen, die auf einem DVB-System basieren}}
\cvline{Betreuer}{Prof. Dr.-Ing. Heinz Gerhäuser, Dipl.-Ing. Christian Forster (LIKE, Uni-Erlangen)}
\cvline{}{Dipl.-Ing. Simon Sudler, Dipl.-Ing. Frank Burkhardt (Fraunhofer IIS)}

%%%%%%%%%%%%%%%%%%
\section{Studienarbeit}
\cvline{Titel}{\emph{Ausarbeitung eines Praktikums zur Vorlesung Architekturen Digitaler Signalverarbeitung}}
\cvline{}{Hardwareentwurf, VHDL-Programmierung, Simulation und
	Inbetriebnahme}
\cvline{Betreuer}{Prof. Dr.techn. Mario Huemer, Dipl.-Ing. Alexander Kölpin}

%%%%%%%%%%%%%%%%%%
\section{Studium}
\cventry{10/03--11/07}{Student}{Friedrich-Alexander-Universität Erlangen-Nürnberg}{Erlangen}{Diplom}{Ingenieur Elektrotechnik (Mikroelektronik)}

%%%%%%%%%%%%%%%%%%
\section{Zivildienst}
\cventry{09/02--08/03}{Zivildienstleistender}{Bayerisches Rotes Kreuz, Kreisverband Main-Spessart}{Gemünden}{Beschädigtenfahrdienst}{}

%%%%%%%%%%%%%%%%%%
\section{Schulbildung \& Ausbildung}
\cventry{09/00--07/02}{Schüler}{Berufsoberschule}{Aschaffenburg}{Fachgebundenes Abitur}{Technischer Zweig}
\cventry{09/97--07/00}{Auszubildender}{Mannesmann Rexorth AG}{Lohr}{Energieelektroniker}{Fachrichtung Betriebstechnik}


\pagebreak

%%%%%%%%%%%%%%%%%%
\section{Sprachkenntnisse}
\cvlanguage{Deutsch}{Muttersprache}{}
\cvlanguage{Englisch}{verhandlungssicher}{}

%%%%%%%%%%%%%%%%%%
\section{EDV-Kenntnisse}
\cvline{Dev-Tools}{JIRA, Confluence, Bitbucket, GIT, Subversion}
\cvline{Office}{MS Office (inkl. VBA), OpenOffice/LibreOffice, LaTeX}
\cvline{SW-Entw.}{Python, (Eclipse PDE, EMF,...), C, C++, PHP, Bash, LabView, u.a.}
\cvline{Embedded}{STMCubeMX/IDE, uVision, PlatformIO, u.a.}
\cvline{E-Tech/CAD}{VHDL (Xilinx ISE, ModelSim, Lattice ispLEVER), SystemC (AMS), KiCAD, Eagle, Target!3001, LTspice, Pspice, FreeCAD, u.a.}
\cvline{Betr.-Syst.}{Windows, Unix (Linux, Solaris), Mac OS X; Embedded: Linux, FreeRTOS}
\cvline{Dienste}{MySQL, Apache, Samba, Maven, Jenkins, OpenVPN, Tomcat, u.a.}
\cvline{Fortbildungen}{Scrum-Master}

%%%%%%%%%%%%%%%%%%
\section{Veröffentlichungen}
\cvline{IUS'12}{\small Simultaneous determination of speed of sound and sample thickness utilizing coded excitation}
\cvline{SENSOR'11}{\small Utilizing CDMA Techniques to Improve Ultrasound Based Distance Measurements}
\cvline{VIP'10}{\small Themenkreis Virtual Test und automatische Testplangenerierung aus ATML}
%\cvline{TuZ'10}{\small Standardized Characterisation of Device Power Supplies for Automated Test Generation and Simulation}
%\cvline{Analog'10}{\small Standardverfahren zur Selbstcharakterisierung und Modellparametrierung von \newline{} Prüflingsspannungsversorgungen in Testsystemen}
\cvline{Analog'10}{\small Standardized Method for Self-Characterization and Model Parametrisation of \newline{} Device Power Supplies in Test Environments}
\cvline{ITC'09}{\small The Best of Both Worlds: Merging the Benefits of Rack\&Stack and Universal ATE}
\cvline{ZuE'09}{\small Increasing Test Quality and Device Reliability by Test Simulation}
\cvline{ZuE'09}{\small Optimales, skalierbares Ressourcenmanagement für \newline{}modulare, gemischt analog-digitale Testsysteme}
\cvline{BMAS'09}{\small Mixed-Signal Test Development using Open Standard Modeling and Description Languages}
\cvline{\footnotesize AUTOTEST'09}{\small Bridge the Gap between Simulation and Test: An OSA-Compliant Virtual Test Environment}
\cvline{DATE'09}{\small A Novel Approach to Entirely Integrate Virtual Test into Test Development Flow}
\cvline{TuZ'09}{\small Vollständige Integration des virtuellen Tests in den Entwurfsprozess integrierter Schaltungen}
\cvline{BLOG}{\small https://www.the78mole.de}
\cvline{Heise Medien}{Freier Autor für Make Magazin}

%%%%%%%%%%%%%%%%%%
\section{Vereine \& Verbände}

\cvline{ZAM}{Zentrum für Austausch und Machen - Betreiberverein ZAM e.V. (Kassenprüfer)}
\cvline{EWERH}{Energiewende ERHlangen e.V. (Vorstand, Schatzmeister)}
\cvline{BUND}{Bund für Umwelt und Naturschutz Deutschland e.V.}
\cvline{VDE}{Verband der Elektrotechnik Elektronik Informationstechnik e.V., \newline{}Informationstechnische Gesellschaft (ITG)}
\cvline{DELUG}{Deutsche Linux User Group}

%%%%%%%%%%%%%%%%%%
\section{Interessen}
\cvline{}{Schaltungsentwurf, Freie Software, 3D-Druck, handwerkliche Tätigkeiten, Heimwerken,}
\cvline{}{Bloggen, Elektromobilität, Fahrrad fahren, Billard, Smart Home, Kochen, Lesen}

% Publications from a BibTeX file without multibib\renewcommand*{\bibliographyitemlabel}{\@biblabel{\arabic{enumiv}}}% for BibTeX numerical labels
%\nocite{*}
%\bibliographystyle{plain}
%\bibliography{publications}       % 'publications' is the name of a BibTeX file

% Publications from a BibTeX file using the multibib package
%\section{Publications}
%\nocitebook{book1,book2}
%\bibliographystylebook{plain}
%\bibliographybook{publications}   % 'publications' is the name of a BibTeX file
%\nocitemisc{misc1,misc2,misc3}
%\bibliographystylemisc{plain}
%\bibliographymisc{publications}   % 'publications' is the name of a BibTeX file

\end{document}


%% end of file `template_en.tex'.
